%% LyX 2.1.2 created this file.  For more info, see http://www.lyx.org/.
%% Do not edit unless you really know what you are doing.
\documentclass{article}
\usepackage[T1]{fontenc}
\usepackage[latin9]{inputenc}
\usepackage{float}

\makeatletter
%%%%%%%%%%%%%%%%%%%%%%%%%%%%%% User specified LaTeX commands.
\usepackage{hyperref}

\makeatother

\begin{document}

\title{Observation and Characterization of Convection Under Static \& Rotating
Conditions and the Possible Implications to the Jovian Atmosphere}


\author{Luke Siemens \& Stephanie Monty}

\maketitle

\section{Introduction}

Current models of Jupiter's atmosphere suggest the possibility of
convection occurring deep within the planet's atmosphere. As early
as 1977 Prinn \& Barshay suggest that the presence of convection within
the atmosphere at 1100K would explain the appearance of carbon monoxide
within the upper atmosphere, as observed in 1977 (Beer \& Taylor,
1978). More recently with observations from the Galileo mission to
Jupiter, convection was inferred from the direct observation of long
lived storms on the planet, appearing as ``zonal jets'' or ``long-lived
ovals'' (Ingersoll et al., 2000). The appearance and longevity of
the storms suggests an energy source within the planet and the transfer
of energy from said source to the atmosphere via convection (Ingersoll
et al., 2000). 

Further support for the presence of convection within the planet can
be found in measuring the effective and equilibrium temperature of
Jupiter. The equilibrium temperature may be found from equating the
incoming solar radiation to the outgoing reradiated solar energy,
while the effective temperature may be found through fitting a black
body spectrum to the integrated flux over all frequencies coming from
the body (Lissauer \& de Pater, 2013). The effective temperature for
Jupiter was measured by the Voyager spacecraft to be $T_{eff}=124.4$K
(Hamel et al., 1981), while the equilibrium temperature can be calculated
to be $T_{eq}=110$K (Mihos, 2005). This discrepency in predicted
and measured temperature alludes to the presence of an internal heat
source within the Jupiter. The emission of this additional energy
from the planet again supports the need for convection in order to
transport the heat from within the planet to it's surface. 

Of interest as well in the Jovian atmosphere is the effect of rotation
on the theorized convection. Jupiter is a rapidly rotating planet
with a rotation rate of approximately 9h 55m (Helled et al., 2009).
This is inferred from the zonal winds in the upper atmosphere, which
can reach \textasciitilde{}100m/s (Helled et al., 2009). Rotation
clearly then has an effect on the motion of Jupiter's atmosphere.
The effect of the rotation could be the introduction of turbulence
which may impact the formation of convection cells and efficiency
of heat transfer via convection. 

With convection being a probable process occurring within the planet,
an attempt to study and become more familiar with the mechanisms associated
with convection in an Earth-based laboratory could yield important
results. An experiment was performed in order to observe and characterize
convection cells on laboratory scales. This included a study into
the generation and characterization of convection under both static
and rotational conditions. The results of this ``table top'' convection
experiment will be discussed in an attempt to relate any findings
to the larger scale of the Jovian interior and atmosphere.


\section{Theory}

The most accessible form of convection to study, characterize and
observe in a laboratory setting, is Rayleigh-Benard (RB) convection.
RB convection is a buoyancy-driven convection which occurs when a
fluid is heated from below and cooled from above. Henri Benard first
observed the creation of hexagonal squares on the surface of a fluid
heated from below in 1900, while Lord Rayleigh derived the theoretical
requirement for the creation of convection within a layer of fluid
``bounded'' by two free surfaces later, in 1916 (Kundu et al.,2012).
Rayleigh showed that a relationship existed between the{*}

As mentioned in the above section, a convective cell will be generated
within a cylindrical vessel and observed through the aid of high speed
cameras, rheoscopic fliuds various dyes. In order to determine the
required temperature difference between the bottom of the vessel and
the top to induce convection, a mathematical study will be made investigating
the Rayleigh-Benard Instability. This will be done using the definition
of the dimensionless Rayleigh Number, noting that the density profile
of the vessel will be treated as a free variable. Working from the
definition of the Rayleigh number.

\[
R_{a}=\frac{g\beta}{\nu\alpha}(T_{b}-T_{u})L^{3}
\]


\[
R_{a}=\frac{g\beta}{\mu k}c_{p}\rho^{2}(T_{b}-T_{u})L^{3}
\]


Noting above that all the variables excluding $T_{b}-T_{u}$ and $\rho$
are fixed constants related to the thermal and viscous characteristics
of the fluid to be used, while $L$ is the height of the vessel and
$g$ is the acceleration due to gravity. The critical Rayleigh number,
$R_{c}$, represents the value of the Rayleigh number, representing
the ratio of gravitational to viscous forces, at which convection
occurs in the system. For the solved case, in which the bottom boundary
is rigid, while the top is free, the case of this experiment, the
critical Rayleigh number is known. Thus, the temperature difference
required to induce convection may then become only dependent on the
density profile of the vessel. If the density if discrete, the temperate
difference may be deduced for each layer of different density fluids,
if the density profile is continuous the temperature difference required
may be expressed in equation (2) where density becomes a function
of height in the vessel. $\rho=\rho(L)$ (Bahrami, 2016).

\begin{equation}
(T_{b}-T_{u})=\left(\frac{\mu k}{g\beta}\right)\frac{R_{c}}{\rho^{2}L^{3}}
\end{equation}


\begin{equation}
(T_{b}-T_{u})=\left(\frac{\mu k}{g\beta}\right)\frac{R_{c}}{\rho^{2}(L)L^{3}}
\end{equation}


The density profile of the vessel will be approximated as being both
linear and discrete as both continuous and discrete density profiles
will be investigated. In reality the more easy of the two to generate
and model will be used in the final experiment to determine the required
temperature difference as given in equations (1) or (2).

A computational simulation will be created in order to investigate
the feasibility and compatibility of both modeling convection in three
dimensions and modeling convection in three dimensions via investigating
convection in two dimensions. This will be done using COMSOL, Python
and possible Fortran for any numerical analysis that may be necessary.
The goal associated with the computational aspect of the project will
be to successfully model convection in two dimensions with the possibility
of extending to three. 

Experimentally, the goal of the project will be to generate and observe
convection cells. Once convection cells have been observed in a stationary
reference frame, rotation will be introduced in an attempt to observe
and record any perturbations to convection that might occur due to
rotation. Any resultant turbulence that could could occur upon the
introduction of rotation into a convective system will also be investigated.
The effects of rotation will only be investigated upon the successful
creation of convection within a stationary reference frame. 

Utilizing the results from the mathematical, computational and experimental
components of the project an overall result as to the possibility,
or inference, of convection within Jupiter's atmosphere will be made.
This will be done in part theoretically, through noting the temperature
difference required in order to generate convection for fluids of
a specific density, or density profile, and comparing this against
the known density profile of the planets atmosphere. Experimentally,
observing convection in the system could lead to inferring the existence
of convection within Jupiter's atmosphere after considering comparable
scale lengths and temperature differences between the two systems.
If the appearance of convection is further supported through the appearance
of convection computationally, via modeling, this will lead to further
confidence in the final result.


\section{References}
\begin{enumerate}
\item Bahrami, M. ``Natural Convection.'' \textit{Simon Fraser University}.
Simon Fraser University, n.d. Web. 21 February 2016.
\item Beer, R. \& Taylor, F. ``The Abundance of Carbon Monoxide in Jupiter.''
\textit{ApJ} 221 (1978): 1100-1109. Print.
\item Ingersoll, A.P. et al. ``Moist convection as an energy source for
the large-scale motions in Jupiter's atmosphere.'' \textit{Nature}
403 (2000): 630-633. Print.
\item Prinn, R.G., Barshay, S.S. ``Carbon Monoxide on Jupiter and Implications
for Atmospheric Convection.'' \textit{AAAS} 198.4321 (1977): 1031-1034.
Print\end{enumerate}

\end{document}
