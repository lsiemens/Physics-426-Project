%% LyX 2.1.2 created this file.  For more info, see http://www.lyx.org/.
%% Do not edit unless you really know what you are doing.
\documentclass{article}
\usepackage[LGR,T1]{fontenc}
\usepackage[latin9]{inputenc}
\usepackage{float}
\usepackage{booktabs}
\usepackage{graphicx}

\makeatletter

%%%%%%%%%%%%%%%%%%%%%%%%%%%%%% LyX specific LaTeX commands.
\DeclareRobustCommand{\greektext}{%
  \fontencoding{LGR}\selectfont\def\encodingdefault{LGR}}
\DeclareRobustCommand{\textgreek}[1]{\leavevmode{\greektext #1}}
\DeclareFontEncoding{LGR}{}{}
\DeclareTextSymbol{\~}{LGR}{126}
%% Because html converters don't know tabularnewline
\providecommand{\tabularnewline}{\\}

%%%%%%%%%%%%%%%%%%%%%%%%%%%%%% User specified LaTeX commands.
\usepackage{hyperref}
\usepackage[title, titletoc, toc, page]{appendix}
\usepackage{listings}

\newcommand{\rawcode}[2]{\subsection*{#2 #1}\label{code:#1}\lstinputlisting[language=Python, breaklines=true, showstringspaces=false]{#1}\vspace{1cm}}

\@ifundefined{showcaptionsetup}{}{%
 \PassOptionsToPackage{caption=false}{subfig}}
\usepackage{subfig}
\makeatother

\begin{document}

\title{Observation and Characterization of Convection Under Static \& Rotating
Conditions and the Possible Implications to the Jovian Atmosphere}


\author{Luke Siemens \& Stephanie Monty}

\maketitle

\section{Introduction}

Current models of Jupiter's atmosphere suggest the possibility of
convection occurring deep within the planet's atmosphere. As early
as 1977 Prinn \& Barshay suggest that the presence of convection within
the atmosphere at 1100K would explain the appearance of carbon monoxide
within the upper atmosphere, as observed in 1977 (Beer \& Taylor,
1978). More recently with observations from the Galileo mission to
Jupiter, convection was inferred from the direct observation of long
lived storms on the planet, appearing as ``zonal jets'' or ``long-lived
ovals'' (Ingersoll et al., 2000). The appearance and longevity of
the storms suggests an energy source within the planet and the transfer
of energy from said source to the atmosphere via convection (Ingersoll
et al., 2000). 

Further support for the presence of convection within the planet can
be found in measuring the effective and equilibrium temperature of
Jupiter. The equilibrium temperature may be found from equating the
incoming solar radiation to the outgoing reradiated solar energy,
while the effective temperature may be found through fitting a black
body spectrum to the integrated flux over all frequencies coming from
the body (Lissauer \& de Pater, 2013). The effective temperature for
Jupiter was measured by the Voyager spacecraft to be $T_{eff}=124.4$ K
(Hamel et al., 1981), while the equilibrium temperature can be calculated
to be $T_{eq}=110$K (Mihos, 2005). This discrepency in predicted
and measured temperature alludes to the presence of an internal heat
source within the Jupiter. The emission of this additional energy
from the planet again supports the need for convection in order to
transport the heat from within the planet to it's surface. 

Of interest as well in the Jovian atmosphere is the effect of rotation
on the theorized convection. Jupiter is a rapidly rotating planet
with a rotation rate of approximately 9h 55m (Helled et al., 2009).
This is inferred from the zonal winds in the upper atmosphere, which
can reach \textasciitilde{}100m/s (Helled et al., 2009). Rotation
clearly then has an effect on the motion of Jupiter's atmosphere.
The effect of the rotation could be the introduction of turbulence
which may impact the formation of convection cells and efficiency
of heat transfer via convection. 

With convection being a probable process occurring within the planet,
an attempt to study and become more familiar with the mechanisms associated
with convection in an Earth-based laboratory could yield important
results. An experiment was performed in order to observe and characterize
convection cells on laboratory scales. This included a study into
the generation and characterization of convection under both static
and rotational conditions. The results of this ``table top'' convection
experiment will be discussed in an attempt to relate any findings
to the larger scale of the Jovian interior and atmosphere.


\section{Theory}

The most accessible form of convection to study, characterize and
observe in a laboratory setting, is Rayleigh-Benard (RB) convection.
RB convection is a buoyancy-driven convection which occurs when a
fluid is heated from below and cooled from above. Henri Benard first
observed the creation of hexagonal squares on the surface of a fluid
heated from below in 1900, while Lord Rayleigh derived the theoretical
requirement for the creation of convection within a layer of fluid
``bounded'' by two free surfaces later, in 1916 (Kundu et al.,2012).
Rayleigh later determined the relationship between that know bears
his name, in the form of the dimensionless Rayleigh number (Ra). The
Rayleigh number may be defined in terms of the following, and expressed
in the following formula:\begin{enumerate}
	\item $\alpha$ = thermal diffusivity
	\item $\beta$ = coefficient of thermal expansion
	\item $\rho$ = density 
	\item $\mu$ = dynamic viscosity
	\item \textit{l} = height of the vessel
	\item \textit{g} = acceleration due to gravity
	\item $\Delta T$ = difference in temperature 
\end{enumerate}
\begin{equation}
R_{a}=\frac{g\beta}{\mu\alpha}\rho(T_{b}-T_{u})l^{3}
\end{equation}


The critical Rayleigh number, $R_{c}$, represents the value of the
Rayleigh number, representing the ratio of gravitational to viscous
forces, at which convection occurs in the system. For the solved case,
in which the bottom boundary is rigid, while the top is free, the
case of this experiment, the critical Rayleigh number is $Ra_{c}=1101$
(Kundu et al.,2012). If a characteristic fluid is chosen such that
all the parameters listed above are fixed, except for the temperature
difference across the fluid, Ra may be deduced on various timescales.
Though, the effect of rotation is not explicitly stated in the calculation
of the Rayleigh number, if one infers the introduction of turbulence
via rotation, and the effect of organization of the fluid flow upon
reaching $Ra_{c}$, one can imagine that perhaps convection occurs
at higher Ra in the rotating case, than the non-rotating case. Thus,
if the temperature difference across the fluid is measured at various
times, the evolution of the Ra number may be studied and information
gained as to the state of the system at said time intervals through
observations of the fluid flow.

Another useful dimensionless number is the Grashof number, a number
that is contained within the definition of the Ra number. The Grashof
number (Gr) reflects the ratio of buoyant to viscous forces (Shire,
2011) and may be given by equation (2). Again, a study as to the evolution
of the ratio of the two forces may be made through a calculation of
Gr at various time steps. Again, the influence of rotation on the
ratio between the two forces may also be inferred through a study
of this type.

\begin{equation}
Gr=\frac{gl^{3}\beta\rho^{2}}{\mu^{2}}(T_{b}-T_{u})
\end{equation}



\section{Experimental Procedure}

A rheoscopic fluid was first created using a mixture of tap water
and fish scales, this was done at a ratio of 1g/200ml consistently
throughout all experiments. In addition, \textasciitilde{}6 drops
of toner fluid were used in order to enhance the effectiveness of
the fluid at detecting bulk motions. In both the static and rotating
case, 400ml of the mixture was placed in a vessel and then on top
of an electronic hot plate. In both the static and rotational case
a sugar mixture was also used to create a discrete density profile.
In this case 100g of sugar was placed on the bottom of the vessel.
Thermocouples were used in order to measure both the temperature of
the fluid, and the temperature of the hot plate at various time steps.
Over a period of \textasciitilde{}40minutes to an hour, measurements
were made of the two temperatures and an image was taken to capture
qualitative observations. This was done for both the static and rotational
cases, in each case \textasciitilde{}10 data points were collected
throughout the experiment. In addition to the procedure mentioned,
during the rotating case the vessel was placed on a rotating table
in which the frequency of rotation was varied from 0-26 RPMs. The
characteristic values associated with the fluid, approximated as pure
water, and vessel may be found below in Table 1.

\begin{table}[H]


\protect\caption{Characteristic Values Used Throughout}


\begin{tabular}{cc}
\toprule 
Variable & Measured or Known Value (Kundu et al., 2012)\tabularnewline
\midrule
\midrule 
$l$ height of vessel & 0.04m\tabularnewline
\midrule 
$\beta$ coefficient of thermal expansion & $2.1\times10^{-4}K^{-1}$\tabularnewline
\midrule 
\textgreek{a} thermal diffusivity & $1.49\times10^{-7}m^{2}/s$\tabularnewline
\midrule 
\textgreek{r} density & $998kg/m^{3}$\tabularnewline
\midrule 
\textgreek{m} dynamic viscosity & $1.002\times10^{-3}kg/ms$\tabularnewline
\bottomrule
\end{tabular}
\end{table}

\subsection{Convection Cell Size}
The characteristic size of convection cells was found using both direct mesurement and comparison to numerical models. In the case where the fluid was suffiently stable the mean size of convections cells was measured directly. In addition to direct mesurement of the convection cell size, the cell size was also determined by fitting the radial Fourier transform of a model of the fluid surface to the observed radial Fourier transform. To find the radial Fourier transform we first used a circular mask to remove the background and normalized the image. Then the 2D Discrete Fast Fourier transform of the image was found and the radial average was computed. This procedure for computing the radial Fourier transform was implimented in the python module \ref{code:radialfft.py}.




\section{Data \& Results}

\begin{figure}[H]
\protect\caption{Experimental Results}


\subfloat[Rotational Convection]{

\includegraphics[scale=0.5]{\string"Rotational_Convection\string".JPG}

}\subfloat[Static Convection]{\includegraphics[scale=0.5]{\string"Static_Convection\string".JPG}

}
\end{figure}
\begin{table}[H]


\protect\caption{Rayleigh and Grashof Results}


\begin{tabular}{ccc}
\toprule 
Case & Maximum Value of Gr & Maximum Value of Ra\tabularnewline
\midrule
\midrule 
Static, No Sugar & $3.54\times10^{6}$ & $2.34\times10^{7}$\tabularnewline
\midrule 
Static, Sugar & $6.79\times10^{6}$ & $4.58\times10^{7}$\tabularnewline
\midrule 
Rotating, No Sugar & $4.95\times10^{6}$ & $3.34\times10^{7}$\tabularnewline
\midrule 
Rotating, Sugar & $8.13\times10^{6}$ & $5.48\times10^{7}$\tabularnewline
\bottomrule
\end{tabular}

\end{table}
\begin{figure}[H]


\protect\caption{Rayleigh Evolution: Static Case}


\includegraphics[scale=0.5]{\string"Static_Rayleigh\string".png}

\end{figure}
\begin{figure}[H]


\protect\caption{Rayleigh Evolution: Static Case Highlighted Turbulent Regime}


\includegraphics[scale=0.5]{\string"Static_Rayleigh_Regime\string".png}

\end{figure}
\begin{figure}[H]


\protect\caption{Rayleigh Evolution: Rotating Case}


\includegraphics[scale=0.5]{\string"Rotating_Rayleigh\string".png}

\end{figure}
\begin{figure}[H]
\protect\caption{Rayleigh Evolution: Rotating Case Highlighted Turbulent and Boiling
Regimes}


\includegraphics[scale=0.5]{\string"Rotating_Rayleigh_Regimes\string".png}

\end{figure}
\begin{figure}[H]


\protect\caption{Grashof Evolution: Static Case}


\includegraphics[scale=0.5]{\string"Static_Grashof\string".png}

\end{figure}
\begin{figure}[H]


\protect\caption{Grashof Evolution: Static Case Highlighted Turbulent Regime}


\includegraphics[scale=0.5]{\string"Static_Grashof_Regime\string".png}

\end{figure}
\begin{figure}[H]
\protect\caption{Grashof Evolution: Rotating Case}


\includegraphics[scale=0.5]{\string"Rotating_Grashof_Regimes\string".png}

\end{figure}
\begin{figure}[H]


\protect\caption{Grashof Evolution: Rotating Case Highlighted Turbulent and Boiling
Regimes}


\includegraphics[scale=0.5]{\string"Rotating_Grashof_Regimes\string".png}

\end{figure}
\begin{figure}[H]


\protect\caption{Cell Size Study, Number of Cells of Each Size (Normalized to the Diameter
of the Vessel)}


%\includegraphics[scale=0.5]{\string"Cell_Numbers\string".png}

\end{figure}
\begin{figure}[H]
\protect\caption{Comparison of Simulated and Experimental Temperature Evolution}


\subfloat[Simulated Results via COMSOL]{

\includegraphics[scale=0.4]{\string"COMSOL\string".JPG}}
\end{figure}
\begin{figure}[H]
\protect\caption{Comparison of Simulated and Experimental Temperature Evolution}


\subfloat[Experimental Results]{



\includegraphics[scale=0.5]{\string"Simulation Comparison\string".png}

}

\end{figure}



\section{Discussion}

The data collected in both the static and rotational case was processed
in a variety of ways. Initially, upon examining the two images of
the convection cells as seen above in both the rotating and static
case, interesting comparisons may be made. In the case of the static
convection, the cells are generally larger than those of the rotating
case. Additionally, the rotating case displays some vorticity within
the cells, visible as curled edges. Right away it can be seen that
rotation has indeed introduced turbulence into the system. Analysis
of the temperature difference across the fluid at various points was
used to calculate both Ra and Gr at every time step and the maximum
values for both numbers, in both cases may be found below in Table
2. The plots of the evolution of each of these may also be found in
the above section. Of note is the definite difference in the values
of the Gr and thus Ra numbers at various times between the non-sugar
and sugar case. The discrete density profile did lead to increased
values of the Ra and Gr in both the rotational and static frames.
This is attributed to the greater difference in temperature across
the fluid. The sugar layer did appear to be efficient at insulating
the bottom of the vessel, thus inhibiting the transmission of heat
across the fluid. Though, it is not accurate obviously to assume a
constant density for use in equation (1), as has been mentioned in
the case of the sugar addition, one was assumed for simplicity in
the calculations of the Ra and Gr numbers. This may have led to artificially
high numbers in the case of the sugar addition, though the high temperature
difference does remain. In all cases, the Ra numbers were well above
$Ra_{c}$inferring the existence of convection, as was observed in
all cases. Of interest was the observation that though convection
was observed throughout the experiment in all cases, static and rotational,
sugar and no-sugar, turbulance was observed at higher temperatures.
Turbulence was observed most notably in the lower layers, often not
permeating to the upper layers. In the case of the sugar addition,
boiling was observed later on in the experiment. The regimes of turbulence
and observed boiling are marked on the plots of Ra and Gr for both
cases, as the green and pink regimes respectively. 

A study was also made into the cell sizes associated with convection
in both static and rotating cases. In the case of the static experiment,
a final image was taken at the end of the experiment (40 minutes)
from which the cell sizes associated with the static convection experiment
were extracted. The average cell size was determined to be 14.70mm.
A plot of the number of cells vs. size may be seen above. 

ADD HERE FOURIER STUFF

A study into simulating the experiment was also made via COMSOL. This
was done using a 2D time-dependent fluid convection model. Water was
chosen as the fluid and a heat flux of $3500W/m^{2}$ was applied to
one of the boundaries. The simulation was completed using a fine mesh,
over 30 minutes, using intervals of 2 minutes. The resultant plot
of temperature of both the top and bottem of the fluid may be seen
above, where the top is given by the blue line and the bottom by the
red. The experimental data for the case of the stationary vessel filled
with rheoscopic fluid mixture (approximated as pure water) can be
seen as well in the above section and may be compared against the
experimental results. It can be said that the shapes of the two temperature
profiles are in good agreement, though the values are in disagreement.

ADD COMPUTATIONAL STUFF

Sources of error within this experiment are numerous as this is only
a preliminary study into the phenomenon of convection, in both static
and rotational reference frames. The approximation of the continuous
density profile used throughout the analysis is an obvious source
of error as it propogates throughout all the calculations. In the
case of this experiment is was deemed too difficult to measure the
change in density profile at different time steps and thus a sacrifice
in accuracy was made. The determination initial cell size was made
by hand and thus itself introduced some errors, though a scale was
set initially within the program via a pixel to physical scale conversion. 

ADD ERRORS

With respect to the observation and study of convection in a laboratory
setting, the experiment may be deemed a success. Convection was observed
fairly easily with the use of the rheoscopic fluid. Beyond that, convection
was observed with only the slight introduction of a temperature difference
across the fluid, owing to the low value of $Ra_{c}$. In the case
of the discrete density profile, the addition of the sugar bottom,
convection, through more turbulent overall, was still observed easily.
This supports the idea that although convection is affected by the
introduction of a discrete density profile, it is not hampered, this
may be analogous to the heavier metallic hydrogen core in Jupiter.
Additionally, rotation, though again introducing more turbulence and
cell deformation, still does not hamper the convection within the
material, this again could be analogous to the evidence of convection
within Jupiter despite the rapid rotation of the planet. 


\section{Conclusion}

Table top convection was observed in both rotating and static reference
frames. Turbulence was observed in both cases, but was much more prominant
in the rotating case, visible most obviously as vorticity in the convection
cell edges. In both kinematic cases, the discrete density profile
led to larger values of Ra and Gr at each time step, though further
investigation is needed to make concrete connections. A COMSOL simulation
was used in 2D to simulate the evolution of the temperature across
the fluid in time and is in good agreement with the experimental results.
The average cell size was also determined via an image and conversion
via a pixel/physical scale for the case of the static frame of reference,
this was found to be 14.70mm. Though further experimentation will
need to be made in order for strong connections between this modest
experiment and the truth of the Jovian atmosphere, it can be said
that rotation does not induce enough turbulence in the laboratory
frame to hinder convection. 


\section{References}
\begin{enumerate}
\item Bahrami, M. ``Natural Convection.'' \textit{Simon Fraser University}.
Simon Fraser University, n.d. Web. 21 February 2016.
\item Beer, R. \& Taylor, F. ``The Abundance of Carbon Monoxide in Jupiter.''
\textit{ApJ} 221 (1978): 1100-1109. Print.
\item Kundu, P.K, Cohen, I.M \& Dowling, D.R. \textit{Fluid Mechanics Fifth Edition}.
Elsevier Inc., 2012. Print.
\item Ingersoll, A.P. et al. ``Moist convection as an energy source for
the large-scale motions in Jupiter's atmosphere.'' \textit{Nature}
403 (2000): 630-633. Print.
\item Lissaur, J. \& de Patar, I. \textit{Fundamentals of Planetary Science}.
Cambridge University Press, 2013. Print
\item Prinn, R.G., Barshay, S.S. ``Carbon Monoxide on Jupiter and Implications
for Atmospheric Convection.'' \textit{AAAS} 198.4321 (1977): 1031-1034.
Print\end{enumerate}

\pagebreak

\begin{appendices}
\section{Code and Data Processing Scripts} \label{app:code}
\rawcode{processdata.py}{Script:}
\rawcode{radialfft.py}{Module:}
\rawcode{voronoi.py}{Module:}

\end{appendices}

\end{document}
